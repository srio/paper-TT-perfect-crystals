%------------------------------------------------------------------------------
% Template file for the submission of papers to IUCr journals in LaTeX2e
% using the iucr document class
% Copyright 1999-2013 International Union of Crystallography
% Version 1.6 (28 March 2013)
%------------------------------------------------------------------------------

\documentclass[preprint]{iucr}              % DO NOT DELETE THIS LINE
\usepackage{bm}
% \usepackage{graphicx}
% \usepackage{tabularx}
% \usepackage{subfigure}
% \usepackage{afterpage}
% \usepackage{sansmath}
\usepackage{mathtools}
% \usepackage{parskip}
% \usepackage{tikz}
% \usepackage{tikzorbital}
% \usepackage{setspace}
% \usepackage{xcolor}
\usepackage{amssymb}
% \usepackage{bm}
\usepackage{amsmath}
% \usepackage{fancyhdr}
% \usepackage{rotating}
\usepackage{siunitx}
\usepackage[hyphens,spaces,obeyspaces]{url}
\usepackage{color}
\usepackage{siunitx}
\usepackage[hyphens,spaces,obeyspaces]{url}
\usepackage{color}
%\usepackage{cprotect}
\usepackage{textgreek}
\usepackage[normalem]{ulem}
\usepackage{makecell}
\usepackage{cancel}
\usepackage{empheq}

\newcommand{\todo}[1]{{\color{red}[TODO: "#1'']}}
\newcommand{\inblue}[1]{{\color{blue}#1}}
\newcommand{\inred}[1]{{\color{red}#1}}
\newcommand{\ingreen}[1]{{\color{green}#1}}



     %-------------------------------------------------------------------------
     % Infobrmation about journal to which submitted
     %-------------------------------------------------------------------------
     \journalcode{S}              % Indicate the journal to which submitted
                                  %   A - Acta Crystallographica Section A
                                  %   B - Acta Crystallographica Section B
                                  %   C - Acta Crystallographica Section C
                                  %   D - Acta Crystallographica Section D
                                  %   E - Acta Crystallographica Section E
                                  %   F - Acta Crystallographica Section F
                                  %   J - Journal of Applied Crystallography
                                  %   M - IUCrJ
                                  %   S - Journal of Synchrotron Radiation

\begin{document}                  % DO NOT DELETE THIS LINE

     %-------------------------------------------------------------------------
     % The introductory (header) part of the paper
     %-------------------------------------------------------------------------

     % The title of the paper. Use \shorttitle to indicate an abbreviated title
     % for use in running heads (you will need to uncomment it).

\title{Calculation of diffraction profiles for perfect crystals obtained as analytical solutions of Takagi-Taupin equations}
% * <msanchezdelrio@gmail.com> 2018-09-25T09:38:50.716Z:
%
% ^.
%\shorttitle{Short Title}

     % Authors' names and addresses. Use \cauthor for the main (contact) author.
     % Use \author for all other authors. Use \aff for authors' affiliations.
     % Use lower-case letters in square brackets to link authors to their
     % affiliations; if there is only one affiliation address, remove the [a].

\cauthor[a]{Jean-Pierre}{Guigay}{guigay@esrf.eu}{address if different from \aff}
\author[a]{Manuel}{Sanchez del Rio}


\aff[a]{European Synchrotron Radiation Facility, 71 Avenue des Martyrs F-38000 Grenoble \country{France}}


     % Use \shortauthor to indicate an abbreviated author list for use in
     % running heads (you will need to uncomment it).

%\shortauthor{Soape, Author and Doe}

     % Use \vita if required to give biographical details (for authors of
     % invited review papers only). Uncomment it.

%\vita{Author's biography}

     % Keywords (required for Journal of Synchrotron Radiation only)
     % Use the \keyword macro for each word or phrase, e.g. 
     % \keyword{X-ray diffraction}\keyword{muscle}

%\keyword{keyword}

     % PDB and NDB reference codes for structures referenced in the article and
     % deposited with the Protein Data Bank and Nucleic Acids Database (Acta
     % Crystallographica Section D). Repeat for each separate structure e.g
     % \PDBref[dethiobiotin synthetase]{1byi} \NDBref[d(G$_4$CGC$_4$)]{ad0002}

%\PDBref[optional name]{refcode}
%\NDBref[optional name]{refcode}

\maketitle                        % DO NOT DELETE THIS LINE

\begin{synopsis}
The Takagi-Taupin equations are solved in its simpler form (zero deformation) and equations of the diffracted and transmitted amplitudes are obtained. Then, the case of a multilayered crystal is discussed using a matrix model. 
\end{synopsis}

\begin{abstract}

The Takagi-Taupin equations are solved in its simpler form (zero deformation) and equations of the diffracted and transmitted amplitudes are obtained. Then, the case of a multilayered crystal is discussed using a matrix model. 

\end{abstract}


     %-------------------------------------------------------------------------
     % The main body of the paper
     %-------------------------------------------------------------------------
     % Now enter the text of the document in multiple \section's, \subsection's
     % and \subsubsection's as required.

\section{Introduction}

The Takagi-Taupin \cite{Takagi1962, Taupin, Taupin1967} equations...

%%%%%%%%%%%%%%%%%%%%%%%%%%%%%%%%%%%%%%%%%%%%%%%%%%%%
%
%%%%%%%%%%%%%%%%%%%%%%%%%%%%%%%%%%%%%%%%%%%%%%%%%%%%
\section{The Takagi-Taupin equations for a plane incident wave}
\label{sec:TT}

Let us start with the Helmholtz equation in a crystal

\begin{equation}
\label{eq:helmholz}
    \Delta \Psi + k^2 (1+\chi) \Psi = 0,
\end{equation}
with $\chi$ the electric susceptibility (refraction index $n=(1+\chi)^{1/2}$) and $\Psi$ the time-independent electric field inside the crystal. In a crystal, the electric susceptibility can be expanded in Fourier series,
\begin{equation}
\label{eq:chi}
    \chi = \sum_{\textbf{h}} \chi_h \exp(i \textbf{h} . \textbf{r}),
\end{equation}
where $\bf{h}$ is the reciprocal lattice vector (with modulus $|\textbf{h}|=2\pi/d_\text{hkl}$, being $d_\text{hkl}$ the d-spacing of the hkl reflection), and the sum goes over all reciprocal vectors $|\textbf{h}|$ with possible hkl Miller indices. If a single reflection $\bf{h}$ is considered, only the terms $\bf{0}$, $\bf{h}$ and $\bf{-h}$ are non-zero, 
\begin{equation}
\label{eq:chisimple}
    \chi = \chi_0 + \chi_{h} \exp(i \textbf{h} . \textbf{r}) + \chi_{-h} \exp(-i \textbf{h} . \textbf{r}).
\end{equation}

The x-ray wavefield inside the crystal is expressed as the sum of two modulated plane waves
\begin{equation}
\label{eq:wavefield}
    \Psi(\textbf r) = D_0(\textbf r) e^{i \textbf k_0 . \textbf r} + D_h(\textbf r) e^{i \textbf k_h . \textbf r},
\end{equation}
with amplitudes $D_{0,h}(\textbf r)$.
The vector $\textbf{k}_h$ that can be defined, without lost of generality, as $\textbf k_h=\textbf k_0 + \textbf h$. 
The wavevector modulus is $k \equiv |\textbf{k}_0|=2\pi/\lambda$, where $\lambda$ is x-ray wavelength in vacuum. Note that, in general, $ |\textbf{k}_h| \ne k$. Only for a direction along the geometrical Bragg position ($\textbf{k}_0=\textbf{k}_0^B$), the Laue equations ($\textbf{k}_h^B=\textbf{k}_0^B+\textbf{h}$) holds, and $|\textbf{k}_0^B|=|\textbf{k}_h^B|$.


% The Takagi-Taupin equations are obtained 
To obtain the Takagi-Taupin equations (TT) we insert the equations~(\ref{eq:chisimple}) and (\ref{eq:wavefield}) in (\ref{eq:helmholz}) and consider two approximations. One supposes $D_{0,h}$ to be slowly varying amplitudes, thus neglecting the 2$^{\text{nd}}$ order derivatives of $D_{0,h}$,  so that
\begin{equation}
\label{eq:approxslowlyvarying}
(\Delta + k^2)[D_{0,h}(\textbf{r}) \exp(i\textbf{k}_{0,h} . \textbf{r})] \approx \exp(i\textbf{k}_{0,h} . \textbf{r}) [2 i \textbf{k}_{0,h} . \nabla D_{0,h} + (k^2 - k^2_{0,h}) D_{0,h}].
\end{equation}
The second approximation neglects two of the six terms in $\chi \Psi$ \todo{justify?}
\begin{subequations}
\label{eq:approxchiPsi}
\begin{align}
\chi\Psi =&
\chi_0 D_0 \exp(i \textbf{k}_0 . \textbf{r}) +
\chi_0 D_h \exp(i \textbf{k}_h . \textbf{r}) +\\
&\chi_h D_0 \exp(i \textbf{k}_h . \textbf{r}) +
\cancelto{\approx 0}{\chi_h D_h \exp(i (\textbf{k}_0+2\textbf{h}) . \textbf{r})} +\\
&\cancelto{\approx 0}{\chi_{-h} D_0 \exp(i (\textbf{k}_0 - 2 \textbf{h}) .\textbf{r})} +
\chi_{-h} D_h \exp(i \textbf{k}_0 . \textbf{r}) .
\end{align}
\end{subequations}
We obtain in this way the TT equations 
\begin{subequations}
\label{eq:TTvector}
\begin{align}
2 i \textbf{k}_0 . \nabla D_0 + \chi_0 k^2 D_0 + \chi_{-h} \inred{k^2} D_h =& 0; \\
2 i \textbf{k}_h . \nabla D_h + (k^2 - k_h^2 + \chi_0 k^2) D_h + \chi_{h} \inred{k^2} D_0 =& 0.
\end{align}
\end{subequations}

We can define a parameter $\alpha$ that measures the deviation of the incident angle from the Bragg position as
\begin{equation}
\label{eq:alpha}
\alpha = \frac{k^2-k_h^2}{k^2} = \frac{k^2-(\textbf k_0 + \textbf h)^2}{k^2} = \frac{\textbf h^2 + 2 \textbf k_0 . \textbf h}{k^2}.
% = 4 \sin \theta_B (\sin \theta - \sin \theta_B) \approx 2 (\theta-\theta_B) \sin 2\theta_B),
\end{equation}

In the ``rotating crystal mode" $\alpha=4 \sin \theta_B (\sin \theta - \sin \theta_B) \approx 2 (\theta-\theta_B) \sin 2\theta_B)$, with $\theta$ the glancing angle on the reflective planes, and $\theta_B$ the Bragg angle ($h=2 k \sin\theta_B$, $\textbf k_0 . \textbf h = -2 k \sin\theta$). Note that the last approximation fails far from the Bragg position or if $\cos\theta_B \rightarrow 0$ (normal incidence). The TT equation become
\begin{subequations}
\label{eq:TTvectorAlpha}
\begin{align}
2 i \textbf{k}_0 . \nabla D_0 + \chi_0 k^2 D_0 + \chi_{-h} \inred{k^2} D_h =& 0; \\
2 i \textbf{k}_h . \nabla D_h + (\alpha + \chi_0) k^2 D_h + \chi_{h} \inred{k^2} D_0 =& 0.
\end{align}
\end{subequations}

A point in the diffraction plane (a plane containing $\textbf{k}_0$ and $\textbf{h}$) 
can be expressed with two oblique coordinates $(s_0,s_h)$ along the directions of the $\textbf k_0$ and $\textbf k_h$ (unit vectors $\hat{ \textbf{s}}_{0}$ and $\hat{ \textbf{s}}_{h}$, respectively). A generic spatial spatial position will also include a third coordinate $\textbf{s}_t$ along an axis $\hat{\textbf{s}}_t=\hat{\textbf{s}}_0 \times \hat{\textbf{s}}_h$, therefore $\textbf r=(s_0,s_h,s_t)$. The director cosines are $\gamma_{0,h,t}=\cos(\textbf{h} . \hat{\textbf{s}}_{0,h,t}) \equiv \cos(\theta_{0,h,t})$. The equation of the crystal surface is $\gamma_0 s_0 + \gamma_h s_h + \gamma_t s_t=0$. For any point $\textbf r=(s_0,s_h,s_t)$ inside the crystal we introduce the path length $s$ \inred{of the incident ray going in this point}: this ray meets the crystal surface in the point of coordinates $(s'_0,s_h,s_t)$ such that $\gamma_0 s'_0+\gamma_h s_h + \gamma_t s_t=0$, so that 
\begin{equation}
\label{eq:s}
s = s_0 - s'_0 = s_0 + s_h \frac{\gamma_h}{\gamma_0} + s_t \frac{\gamma_t}{\gamma_0}.
\end{equation}

The simple relation $d s_0 = \nabla s_0 . [ d s_0 . \hat{\textbf{s}}_0 + d s_h \hat{\textbf{s}}_h + d s_t \hat{\textbf{s}}_t ]$ implies $\nabla s_0 . \hat{\textbf{s}}_0=1$ and $\nabla s_0 . \hat{\textbf{s}}_{h,t}=0$. Similarly, $\nabla s_h . \hat{\textbf{s}}_h=1$ and $\nabla s_h . \hat{\textbf{s}}_{0,t}=0$. Therefore, 
\begin{subequations}
\label{eq:equalities}
\begin{align}
\hat s_0 . \nabla D=
\hat s_0 . \left[ 
\frac{\partial D}{\partial s_0} \nabla s_0 + 
\frac{\partial D}{\partial s_h} \nabla s_h +
\frac{\partial D}{\partial s_t} \nabla s_t
\right] 
=& \frac{\partial D}{\partial s_0}
; \\
\hat s_h \nabla D =& 
\frac{\partial D}{\partial s_h}.
\end{align}
\end{subequations}

Using the approximation \todo{physical meaning, justify?}
\begin{equation}
\label{eq:approxKH}
\textbf k_h . \nabla D_h = |\textbf k_h| \frac{\partial D_h}{\partial s_h} \approx k \frac{\partial D_h}{ \partial s_h},
\end{equation}
we obtain from equations~(\ref{eq:TTvector})
\begin{subequations}
% \label{eq:TT}
% \begin{align}
\begin{empheq}[box=\fbox]{align}
\frac{\partial D_0}{\partial s_0} =& \frac{ik}{2} \left[ \chi_0 D_0(s_0,s_h,s_t)+ \chi_{-h} D_h(s_0,s_h) \right]; \\
\frac{\partial D_h}{\partial s_h} =& \frac{ik}{2} \left[ (\chi_0 + \alpha) D_h(s_0,s_h,s_t)+ \chi_{h} D_0(s_0,s_h,s_t) \right],
% \end{align}
\end{empheq}
\end{subequations}
or in a more compact form
\begin{subequations}
\label{eq:TTcompact}
\begin{align}
\frac{\partial D_0}{\partial s_0} =& i u_0 D_0(s_0,s_h,s_t) + i u_{-h} D_h(s_0,s_h,s_t); \\
\frac{\partial D_h}{\partial s_h} =& i (u_0 + \alpha') D_h(s_0,s_h,s_t) + i u_{h} D_0(s_0,s_h,s_t),
\end{align}
\end{subequations}
where it has been used the notation $u_{0,h,-h}=(\lambda/\pi) \chi_{0,h,-h}$ and $\alpha' = (\lambda/\pi) \alpha$.



%%%%%%%%%%%%%%%%%%%%%%%%%%%%%%%%%%%%%%%%%%%%%%%%%%%%
%
%%%%%%%%%%%%%%%%%%%%%%%%%%%%%%%%%%%%%%%%%%%%%%%%%%%%
\section{Solutions of TT equations for perfect crystals}
\label{sec:TTsolutions}

%%%%%%%%%%%%%%%%%%%%%%%%%%%%%%%%%%%%%%%%%%%%%%%%%%%%
%
%%%%%%%%%%%%%%%%%%%%%%%%%%%%%%%%%%%%%%%%%%%%%%%%%%%%

It is interesting to consider first the effects of refraction and absorption without Bragg diffraction. 
According to equation~(\ref{eq:TTcompact}a), the differential equation for the refracted wave is
\begin{equation}
\frac{\partial D_0^{\text{ref}}}{\partial s_0} = i u_0 D_0^{\text{ref}};
\end{equation}
its solutions ate of the form $D_0^{\text{ref}}=f(s_h) \exp(i u_0 s_0)$, and the boundary condition is $D_0^{\text{ref}}=1$ for  $\gamma_0 s_0 + \gamma_h s_h + \gamma_t s_t=0$, from which $f(s_h,s_t)=\exp(i u_0 (s_h \gamma_h + s_t \gamma_t)/\gamma_0)$; therefore $D_0^{\text{ref}}= \exp(i u_0 s_0 + i u_0(s_h \gamma_h + s_t \gamma_t)/\gamma_0) \equiv \exp(i u_0 s)$, with $s=s_0+(s_h \gamma_h+s_t \gamma_t)/\gamma_0$.

Consequently, we will consider solutions of equations (\ref{eq:TTcompact}) depending on the single variable $s$, which means 
$D'_{0}(s)=\partial D_{0} / \partial s_{0}$ and $D'_{h}(s)\gamma_h/\gamma_0=\partial D_{h} / \partial s_{h}$,
% We define $b=\gamma_h/\gamma_0$,  ($\theta_{0,h}$ the angles of $\textbf{k}_{0,h}$ with respect to the inward normal to the crystal surface) and $s=s_0+s_h/b=z/\cos \theta_0$ (the crystal surface has equation $z=s=0$). 
the equations~(\ref{eq:TTcompact}) are become
\begin{subequations}
\label{eq:TTlaue}
\begin{align}
D'_0(s) =& i u_0 D_0(s) + i u_{-h} D_h(s); \\
D'_h(s) =& i b (u_0 + \alpha') D_h(s) + i b u_{h} D_0(s).
\end{align}
\end{subequations}

It is convenient to introduce the functions $B_{0,h}(s)$ by setting
\begin{equation}
\label{eq:Bdefinition}
D_{0,h} = \exp \left( i s \frac{u_0 + b (u_0+\alpha')}{2} \right) B_{0,h} = \exp(i s (u_0+\omega)) B_{0,h},  
\end{equation}
with $\omega=(b(u_0+\alpha')-u_0)/2$. They are solutions of 
\begin{subequations}
\label{eq:TTinB}
\begin{align}
B'_0(s) =& -i \omega B_0(s) + i u_{-h} B_h(s); \\
B'_h(s) =& i \omega B_h(s) + i b u_{h} B_0(s).
\end{align}
\end{subequations}


% \inred{Let us call $\textbf{q}$ the wavevector inside the crystal produced by the the incident wave $\textbf{k}_0$ (refracted). $\Psi=\exp(i q s_0)$, that introduced in equation (\ref{eq:helmholz}) results $q^2=k^2(1+\chi)\approx k (1+\chi_0/2)$. Therefore $\exp(i q s_0) = \exp(i (k/2) (1+Re(\chi_0)) s_0) \times \exp(-(k/2) Im(\chi_0) s_0)$. Being the second factor the absorption, we must have $Im(\chi_0)>0$.
% }

\inblue{Note that $\omega=(\pi/\lambda) w$, with $w=b \sin(2\theta_B)(\theta-\theta_B)+\chi_0 (b-1)/2=b \sin(2\theta_B)(\theta-\theta_c+ i [(b-1)/2] \operatorname{Im} \chi_0 )$, where $\theta_c=\theta_B+ (b-1)/(b \sin(2\theta_B)) \operatorname{Re}\chi_0$ is the corrected Bragg angle.}

Looking for solutions of equation~(\ref{eq:TTinB}) of the form $B_0=\exp(i \eta s)$, $B_h=\xi \exp(i \eta s)$, we obtain from equations (\ref{eq:TTinB}): $\eta =-\omega + \xi u_{-h}$, $\xi \eta=\xi \omega+b u_h$,
hence $\xi=(\eta+\omega)/u_{-h}=b u_h / (\eta-\omega)$, $\eta^2-\omega^2=b u_h u_{-h}$ with solutions $\eta_{1,2}=\pm a$, with $a=\sqrt{b u_h u_{-h}+\omega^2}$.

%%%%%%%%%%%%%%%%%%%%%%%%%%%%%%%%%%%%%%%%%%%%%%%%%%%%
%
%%%%%%%%%%%%%%%%%%%%%%%%%%%%%%%%%%%%%%%%%%%%%%%%%%%%
\subsection{Laue case}
\label{sec:TTsolutionsLaue}

For the Laue case $b>0$. The solution can be obtained in two ways:

\subsubsection{Laue solution (general)}
The solutions of equation (\ref{eq:TTinB}) can be written as
\begin{subequations}
\label{eq:TTlaueSolutions}
\begin{align}
B_0 = c_1 \exp(i a s) + c_2 \exp(-i a s) =& (c_1+c_2) \cos(as) + (c_1-c_2) i \sin(as); \\
B_h = c_1 \xi_1 \exp(i a s) + c_2 \xi_2 \exp(-i a s) =& (c_1 \xi_1+c_2 \xi_2) \cos(as) + (c_1 \xi_1-c_2 \xi_2) i \sin(as).
\end{align}
\end{subequations}

The coefficients $c_{1,2}$ are calculated from the boundary conditions $c_1+c_2=1$ and $c_1\xi_1+c_2\xi_2=0$ as follows
\begin{subequations}
\label{eq:TTlaueCoefficients}
\begin{align}
c_1=\frac{\xi_2}{\xi_2-\xi_1}, c_2=\frac{-\xi_1}{\xi_2-\xi_1}  & \text{~~with~~} \xi_{1,2} = \frac{\omega \pm a}{u_{-h}} \text{,~~or} \\
\xi_1-\xi_2 = \frac{2 a}{u_{-h}} \text{~~and~~} & \xi_1+\xi_2 = \frac{2 \omega}{u_{-h}} .
\end{align}
\end{subequations}
Therefore 
\begin{subequations}
\label{eq:TTlaueCoefficients2}
\begin{align}
&c_1-c_2=\frac{\xi_1+\xi_2}{\xi_2-\xi_1}=-\frac{\omega}{a}\\
&c_1\xi_1-c_2\xi_2=\frac{2\xi_1\xi_2}{\xi_2-\xi_1}=
2\frac{\omega^2-a^2}{u_{-h}^2}\frac{u_{-h}}{-2a}=\frac{-b u_h u_{-h}}{-a u_{-h}}=\frac{b u_h}{a}.
\end{align}
\end{subequations}
Finally
\begin{subequations}
\label{eq:laueSolutionsB}
\begin{align}
B_0 = & \cos(a s) - i \omega \sin(a s) / a \\
B_h = & i b u_h \frac{\sin(a s)}{a} = \frac{i b \chi_h}{\sqrt{b \chi_h \chi_{-h} + w^2}} \sin \left( s \frac{\pi}{\lambda} \sqrt{b \chi_h \chi_{-h} + w^2}\right),
\end{align}
\end{subequations}
\inred{
which in terms of $D$
\begin{subequations}
\label{eq:laueSolutions}
\begin{align}
D_0 = & \left(\cos(a s) - i \frac{\omega}{a} \sin(a s) \right) \exp(i s u_0+i s \omega) \\
D_h = & i  \frac{b u_h}{a}\sin(a s) \exp(i s u_0+i s \omega).
\end{align}
\end{subequations}
}

The intrinsic diffraction profile (often referred as rocking curve) is obtained from equations~(\ref{eq:laueSolutions}) as $|D_h(\theta)|^2=|B_h(\theta)|^2$, taking into account the expression of $w$ and $s=z/\cos(\theta_0)$ at $z=t$ with $t$ the thickness of the crystal. \todo{add power factor}

\subsubsection{Laue solution based on Laplace transform}
\label{sec:laplaceLaue}
Let denote $\bar{F}(p)$ the Laplace transform of a function $F(s)$
\begin{equation}
\Bar{F}(p) = \int_0^\infty ds \exp(-p s) F(s).
\end{equation}
Applying the Laplace transform to equations~(\ref{eq:TTinB}) we get
\begin{subequations}
\label{eq:TTlaueLaplace}
\begin{align}
(p + i \omega) \bar{B_0}(p) - i u_{-h} \bar{B_h}(p)= & 1 \\
(p - i \omega) \bar{B_h}(p) - i u_{h} \bar{B_0}(p)= & 0.
\end{align}
\end{subequations}
The solutions are
\begin{subequations}
\begin{align}
\bar{B_0}(p) &= \frac{(p - i \omega) }{p^2 + a^2} \\
\bar{B_h}(p) &= \frac{i b u_h}{p^2 + a^2},
\end{align}
\end{subequations}
with $a^2=\omega^2 + b u_h u_{-h}$, $a=\sqrt{\omega^2+b u_h u_{-h}}$
hence the above results, $(p^2+a^2)^{-1})$ and $p(p^2+a^2)^{-1}$ being the Laplace transform of $\cos(a s)$ and $\sin(a s)/a$, respectively. 

Using the formula $\sin(p+i q)=\sin p \cos(i q) + \cos p \sin(i q)=\sin p \cosh q + i \cos p \sinh q$. it is shown that $|\sin(a s)|^2=\sin^2(s \operatorname{Re}(a)) + \sinh^2(s \operatorname{Im}(a)$;

%%%%%%%%%%%%%%%%%%%%%%%%%%%%%%%%%%%%%%%%%%%%%%%%%%%%
%
%%%%%%%%%%%%%%%%%%%%%%%%%%%%%%%%%%%%%%%%%%%%%%%%%%%%
\subsection{Bragg case}
\label{sec:TTsolutionsBragg}

For the Bragg case $b<0$. Considering $T$ the path length of the incident beam in the crystal, and $r=B_h(0)$, the boundary conditions are $B_0(0)-1$ and $B_h(T)=0$. By Laplace transform of equation~(\ref{eq:TTinB}), we obtain
\begin{subequations}
\label{eq:TTbraggLaplace}
\begin{align}
(p + i \omega) \bar{B_0}(p) - i u_{-h} \bar{B_h}(p)= & 1 \\
(p - i \omega) \bar{B_h}(p) - i u_{h} \bar{B_0}(p)= & r,
\end{align}
\end{subequations}
or 
\begin{subequations}
\begin{align}
\bar{B_0}(p) &= \frac{p - i \omega + i r u_{-h}}{p^2 + a^2} \\
\bar{B_h}(p) &= \frac{r (p + i \omega) + i b u_h}{p^2 + a^2},
\end{align}
\end{subequations}
with $a^2=\omega^2+b u_h u_{-h}$. Hence:
\begin{subequations}
\begin{align}
B_0(s) &= \cos(a s) + i (r u_{-h} - \omega) \frac{\sin(a s)}{a}\\
B_h(s) &= r [\cos(a s) + i (\omega/a) \sin(a s)] + i b u_h \frac{\sin(a s)}{a}.
\end{align}
\end{subequations}

The reflected amplitude $r$ and transmitted amplitude corrsponding at $B_0(T)$ are obtained using the condition $B_h(T)=0$. With some calculation, we obtain: 
\begin{subequations}
\begin{align}
r &= \frac{-i b u_h \sin(a T)}{a \cos(a T) + i\omega \sin(a T)}\\
B_0(T) &= \frac{a}{a \cos(a T) + i\omega \sin(a T)},
\end{align}
\end{subequations}
with $a=\sqrt{\omega^2 + b u_h u_{-h}}$. 
\todo{amplitudes for $D$}

\subsubsection{Case with no absorption}
$a^2$ is real negative if $|\theta-\theta_c|<|\chi_h|/(|b|^2 \sin(2 \theta_B))$, then $aT=i \operatorname{Im}(a) T$, $\cos(a T)=\cosh(\operatorname{Im}(a)T)$, $\sin(a T)=i \sinh(\operatorname{Re}(a)T)$, 
\begin{equation}
    r = \frac{b u_h \sinh(\operatorname{Im}(a)T)}{i \operatorname{Im}(a)T \cosh(\operatorname{Im}(a)T)- \omega
    \sinh(\operatorname{Im}(a)T)}=
\frac{-i b u_h \tanh(\operatorname{Im}(a)T)}{\operatorname{Im}(a)+ i \omega \tanh(\operatorname{Im}(a)T)}.
\end{equation}
For $t \rightarrow \infty$, 
\begin{equation}
    r \rightarrow - \frac{i b u_h}{\operatorname{Im}(a)+ i \omega} b \sin(2 \theta_B)(\theta-\theta_c)+ i \frac{b-1}{2}\operatorname{Im}(\chi_0).
\end{equation}



% We define $b=\sin\theta_0/\sin\theta_h$ ($\theta_{0,h}$ the grancing angles of the incident and reflected directions with respect to the crystal surface, and $s=s_0+s_h/b$. Unlike the usual convention, we have $b>0$. The crystal surface has equation $s=0$. 

% The refractive wave is such that $D_0^{\text{ref}}=\exp(u_0 s_0) f(s_h)$, with $D_0^{\text{ref}}(s_h/b,s_h)=1$; therefore $\exp(i u_0 s_h/b)=1$ and $D_0^{\text{ref}}=\exp(i u_0 (s_0 - s_h/b) )=\exp(i u_o s)$; this implies $Im(u_0)>0$.


% The TT equations are then
% \begin{subequations}
% \label{eq:TTlaue}
% \begin{align}
% D'_0(s) =& i u_0 D_0(s) + i u_{-h} D_h(s); \\
% D'_h(s) =& -i b (u_0 + \alpha') D_h(s) - i b u_{h} D_0(s).
% \end{align}
% \end{subequations}

% We introduce the functions $B_{0,h}(s)$ by setting
% \begin{equation}
% \label{eq:TTbraggB}
% D_{0,h} = \exp \left( i s \frac{u_0 - b (u_0+\alpha')}{2} \right) B_{0,h} = \exp(i s (u_0-\omega)) B_{0,h},  
% \end{equation}
% with $\omega=(b(u_0+\alpha')+u_0))/2$. 

% \inblue{Note that $\exp(i s (u_0 - \omega))=\exp(-s(1+b)/2+ b \alpha/2))=\exp(-s(1+b)/2+ b \sin(2 \theta_B) (\theta - \theta_c)))$, with the corrected value of the Bragg angle $\theta_c=\theta_s - \frac{(1+b) Re(\chi_0)}{2 b \sin(2 \theta_B)}$
% }



% They are solutions of 
% \begin{subequations}
% \begin{align}
% B'_0(s) =& -i \omega B_0(s) + i u_{-h} B_h(s); \\
% B'_h(s) =& i \omega B_h(s) + i b u_{h} B_0(s).
% \end{align}
% \end{subequations}



%%%%%%%%%%%%%%%%%%%%%%%%%%%%%%%%%%%%%%%%%%%%%%%%%%%%
%
%%%%%%%%%%%%%%%%%%%%%%%%%%%%%%%%%%%%%%%%%%%%%%%%%%%%
\section{Conclusions and future perspectives}
\label{sec:summary}



% \ack{Acknowledgements}
% A large part of this work comes from ...


\bibliography{iucr} % reads iucr.bib with items
\bibliographystyle{iucr}
%\referencelist{library}

% \appendix
% \section{Derivation of the Lens Equation from the phase-factor of the Takagi-Taupin equations}
% \label{appendix:CLE}





     %-------------------------------------------------------------------------
     % The back matter of the paper - acknowledgements and references
     %-------------------------------------------------------------------------

     % Acknowledgements come after the appendices



     % References are at the end of the document, between \begin{references}
     % and \end{references} tags. Each reference is in a \reference entry.

%\begin{references}
%\reference{Author, A. \& Author, B. (1984). \emph{Journal} \textbf{Vol}, first page--last page.}
%\end{references}
%\cite{knuth84}

%\begin{thebibliography}{30}
%\expandafter\ifx\csname natexlab\endcsname\relax\def\natexlab#1{#1}\fi
%\expandafter\ifx\csname bibnamefont\endcsname\relax
%  \def\bibnamefont#1{#1}\fi
%\expandafter\ifx\csname bibfnamefont\endcsname\relax
%  \def\bibfnamefont#1{#1}\fi
%\expandafter\ifx\csname citenamefont\endcsname\relax
%  \def\citenamefont#1{#1}\fi
%\expandafter\ifx\csname url\endcsname\relax
%  \def\url#1{\texttt{#1}}\fi
%\expandafter\ifx\csname urlprefix\endcsname\relax\def\urlprefix{URL }\fi
%\providecommand{\bibinfo}[2]{#2}
%\providecommand{\eprint}[2][]{\url{#2}}
%
%\bibitem[{\citenamefont{Shvyd'ko}(2016)}]{GuigayFerrero2013}
%\bibinfo{author}{\bibfnamefont{Yu.}~\bibnamefont{Shvyd'ko}},
%\bibinfo{journal}{Phys. Rev. Lett.} \textbf{\bibinfo{volume}{116}},
%\bibinfo{pages}{080801} (\bibinfo{year}{2016}).
%
%%\bibitem[{\citenamefont{Guigay and Ferrero}]{ddd}
%%\bibinfo{author}{\bibfnamefont{Jean-Pierre}~\bibnamefont{Guigay}},
%%\bibinfo{author}{\bibfnamefont{Claudio}~\bibnamefont{Ferrero}},
% % \bibinfo{journal}{xxxx} \textbf{\bibinfo{volume}{xx}},
% % \bibinfo{pages}{xx} (\bibinfo{year}{2012}).
%
%\end{thebibliography}

%% Note added by Overleaf: If using bibtex, remove the "references" environment above, and uncomment the following lines.

%\bibliographystyle{iucr}
%\referencelist{iucr}

%      %-------------------------------------------------------------------------
%      % TABLES AND FIGURES SHOULD BE INSERTED AFTER THE MAIN BODY OF THE TEXT
%      %-------------------------------------------------------------------------
% 
%      % Simple tables should use the tabular environment according to this
%      % model
% 
% \begin{table}
% \caption{Caption to table}
% \begin{tabular}{llcr}      % Alignment for each cell: l=left, c=center, r=right
%  HEADING    & FOR        & EACH       & COLUMN     \\
% \hline
%  entry      & entry      & entry      & entry      \\
%  entry      & entry      & entry      & entry      \\
%  entry      & entry      & entry      & entry      \\
% \end{tabular}
% \end{table}
% 
%      % Postscript figures can be included with multiple figure blocks
% 
% \begin{figure}
% \caption{Caption describing figure.}
% \includegraphics{fig1}
% \end{figure}


\end{document}                    % DO NOT DELETE THIS LINE
%%%%%%%%%%%%%%%%%%%%%%%%%%%%%%%%%%%%%%%%%%%%%%%%%%%%%%%%%%%%%%%%%%%%%%%%%%%%%%
